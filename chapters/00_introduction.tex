\section{Introduction}

Strategic card games have long provided a natural framework for the study of sequential decision-making, probabilistic modeling, and zero-sum interactions. Their combinatorial structure, partial information, and payoff mechanisms make them ideal objects for analyzing rational behavior in uncertain environments. Despite this interest, scientific literature remains fragmented: the majority of studies focus either on combinatorial aspects, transition dynamics, or payoff functions, without offering a unified framework capable of simultaneously integrating these dimensions.

This lack of complete formalization is particularly notable for fixed-structure sequential games, where initial distribution, inter-round dependencies, and non-linear payoff mechanisms interact in complex ways. To date, few models offer a coherent mathematical representation allowing for the joint study of combinatorial structure, stochastic dynamics, and strategic balancing properties.

This work is situated within this context and introduces the \emph{F5 Game} (KSZ Five-Five Card Model), a finite game of imperfect information consisting of five successive rounds and incorporating a multiplicative payoff system. \textbf{This paper introduces a new formally axiomatized game model rather than a survey of existing games.} Our objective is to propose a rigorous formalization of this model and to formally establish the resulting structural, probabilistic, and strategic properties. To achieve this, we develop the \emph{Kaptue‑F5 Law}, a hierarchical probabilistic framework combining multivariate hypergeometric distribution, non-stationary Markov processes, and deterministic payoff functions.

The primary contributions of this work are as follows:
\begin{itemize}
    \item We provide a complete axiomatic definition of the game, including the state space, legal actions, transitions, and victory conditions;
    \item We introduce the Kaptue‑F5 Law, enabling a unified modeling of hand distribution, sequential dynamics, and payoff mechanisms;
    \item We formally establish fundamental structural properties, such as winner uniqueness, stake conservation, and the finiteness of the game tree;
    \item We present a detailed probabilistic analysis including distributions associated with card sums, asymptotic results, and the evaluation of key probabilities;
    \item We conduct a study of balancing mechanisms, specifically determining the optimal multiplier for the Cora system and analyzing payoff variance.
\end{itemize}

The structure of the document is as follows.
Section~1 presents the general context of the F5 Game.
Section~2 introduces formal definitions and the axiomatic system.
Section~3 establishes fundamental structural properties.
Section~4 analyzes combinatorial aspects and the finiteness of the game tree.
Section~5 develops advanced probabilistic analysis.
Section~6 studies balancing mechanisms and the Cora system.
Section~7 addresses algorithmic complexity.
Section~8 presents the Kaptue‑F5 Law and its extensions.
Finally, Section~9 proposes simulations and computational applications.

\paragraph{Conclusion}
In summary, this introduction has highlighted the necessity of a unified mathematical framework for the study of sequential games with imperfect information. The F5 Game constitutes a model particularly suited for this analysis, and the Kaptue‑F5 Law offers a coherent probabilistic structure to study its properties. The following sections successively develop the formalization of the game, the analysis of its fundamental properties, and the theoretical tools required to understand its dynamics and balancing mechanisms.