% ========================================
% CHAPTER 7: COMPLEXITY THEOREMS OF THE F5 GAME
% ========================================

\section{Complexity Theorems of the \textbf{F5 Game}}
\label{chap:complexity_theorems}

This chapter analyzes the structural and algorithmic complexity of the \textbf{F5 Game}. We successively study:
(1) the size of the game tree,
(2) the temporal and spatial complexity of a complete simulation,
(3) the complexity of the dealing procedure.

% ========================================
\subsection{Game Tree Complexity}
\label{subsec:game_tree_complexity}

\begin{theorembox}{Upper bound on game tree size}{theo:upper_bound_tree}
\label{theo:upper_bound_tree}
We define the maximum number of nodes in the game tree as being upper-bounded by:
\[
N_{\text{nodes}} \leq 4^5 \times (5!)^n.
\]
\end{theorembox}

\begin{proof}
At each round $r$:
\begin{itemize}
    \item the controller chooses a suit from 4 options;
    \item each player has $(6-r)$ remaining cards, thus at most $(6-r)$ choices.
\end{itemize}

Over 5 rounds:
\[
4^5 = 1024
\]

and each player performs a permutation of their 5 cards:
\[
5! = 120.
\]

For $n$ players:
\[
N_{\text{nodes}} \leq 1024 \times (120)^n.
\]

For $n=4$:
\[
N_{\text{nodes}} \leq 1024 \times 120^4 \approx 2.12 \times 10^{11}.
\]
\end{proof}

\begin{theorembox}{Tree depth}{theo:tree_depth}
\label{theo:tree_depth}
The depth of the game tree is exactly:
\[
d = R = 5.
\]
\end{theorembox}

\begin{proof}
A game consists of exactly 5 rounds, each corresponding to a decision level.
No additional branching is possible: we therefore conclude that the depth is strictly equal to 5.
\end{proof}

% ========================================
\subsection{Algorithmic Complexity}
\label{subsec:algorithmic_complexity}

\begin{theorembox}{Time complexity of a full simulation}{theo:time_complexity_simulation}
\label{theo:time_complexity_simulation}
The time complexity for a complete simulation of a game session is:
\[
\mathcal{O}(n \times R) = \mathcal{O}(5n).
\]
\end{theorembox}

\begin{proof}
In each round:
\begin{itemize}
    \item choice of suit: $\mathcal{O}(1)$;
    \item cards played by $n$ players: $\mathcal{O}(n)$;
    \item winner determination: $\mathcal{O}(n)$.
\end{itemize}

Thus, one round costs $\mathcal{O}(n)$, and for $R=5$ rounds:
\[
\mathcal{O}(n) \times 5 = \mathcal{O}(5n).
\]
\end{proof}

\begin{theorembox}{Space complexity}{theo:space_complexity}
\label{theo:space_complexity}
For fixed deck size $|D| = 32$ and fixed maximum number of players $n \leq 4$, the space complexity required to store a game state is:
\[
\mathcal{O}(1).
\]
\end{theorembox}

\begin{proof}
A state contains:
\[
G(t) = (H(t), r(t), c(t), s_r, P(t), \sigma, \text{direction}, d, M_0, \text{Disq}(t)).
\]

All these objects have a size bounded by constants independent of the input, as both $n \leq 4$ and $|D| = 32$ are fixed constants.
The required memory is therefore constant.
\end{proof}

% ========================================
\subsection{Complexity of the Dealing Procedure}
\label{subsec:dealing_complexity}

\begin{theorembox}{Complexity of the dealing procedure}{theo:dealing_complexity}
\label{theo:dealing_complexity}
For fixed deck size $|D| = 32$, the time complexity of the complete dealing procedure (shuffling, cutting, distribution) is \cite{cormen2009introduction, sipser2012introduction, fisher1934randomisation}:
\[
\mathcal{O}(|D| \log |D|) = \mathcal{O}(32 \log 32) = \mathcal{O}(1).
\]
\end{theorembox}

\begin{proof}
\begin{itemize}
    \item Shuffling (Fisher-Yates): $\mathcal{O}(32)$;
    \item Cutting: $\mathcal{O}(1)$;
    \item Distribution: $\mathcal{O}(5n) = \mathcal{O}(20)$.
\end{itemize}

Thus, the total complexity is:
\[
\mathcal{O}(32 + 1 + 20) = \mathcal{O}(32) = \mathcal{O}(1).
\]

Since the deck size is fixed at 32 cards, all operations require constant time.
\end{proof}

\begin{notebox}{Constant complexity of dealing}{note:constant_dealing_complexity}
\label{note:constant_dealing_complexity}
We observe that the distribution is asymptotically constant (for fixed deck size), which allows for the implementation of massive game simulations without significant algorithmic overhead.
\end{notebox}

\begin{theorembox}{Controlled algorithmic complexity}{theo:complexity_summary}
\label{theo:complexity_summary}
The \textbf{F5 Game} possesses controlled algorithmic complexity in the following sense:
\begin{enumerate}
    \item the depth of the game tree is constant ($R=5$);
    \item the time complexity of a full simulation is linear with respect to the number of players: $\mathcal{O}(5n)$;
    \item the space complexity of a game state is constant (for fixed deck size): $\mathcal{O}(1)$;
    \item the dealing procedure has an asymptotically constant complexity (for fixed deck size).
\end{enumerate}
\end{theorembox}

\begin{proof}
\begin{enumerate}
    \item Follows from Theorem~\ref{theo:tree_depth}.
    \item Follows from Theorem~\ref{theo:time_complexity_simulation}.
    \item Follows from Theorem~\ref{theo:space_complexity}.
    \item Follows from Theorem~\ref{theo:dealing_complexity}.
\end{enumerate}
Thus, all essential components of the game have bounded or linear complexity, ensuring full control over the algorithmic cost.
\end{proof}