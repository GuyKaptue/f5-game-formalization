% ========================================
% CHAPTER 5: ADVANCED PROBABILISTIC ANALYSIS
% ========================================

\section{Advanced Probabilistic Analysis of the \textbf{F5 Game}}
\label{chap:probabilistic_analysis}

This chapter studies the probabilistic properties of the \textbf{F5 Game}, specifically the distribution of hand values, the probabilities associated with rare events (Cora, triple 7) \cite{ross2014first}, and the combinatorial structures underlying card distribution.

% ========================================
\subsection{Distribution of the Hand Sum}
\label{subsec:sum_distribution}

\begin{theorembox}{Expectation of the hand sum}{theo:sum_expectation}
\label{theo:sum_expectation}
For a random hand $h$ of five cards drawn without replacement from the full deck, we find that the expected sum is:
\[
\mathbb{E}[\Sigma(h)] = 5 \cdot \mu_V = 32.5.
\]
\end{theorembox}

\begin{proof}
By the linearity of expectation:
\[
\mathbb{E}[\Sigma(h)]
= \mathbb{E}\left[\sum_{i=1}^{5} \text{val}(c_i)\right]
= \sum_{i=1}^{5} \mathbb{E}[\text{val}(c_i)].
\]
Each card possesses the same marginal distribution (sampling without replacement):
\[
\mathbb{E}[\text{val}(c_i)] = \mu_V = 6.5.
\]
Hence:
\[
\mathbb{E}[\Sigma(h)] = 5 \times 6.5 = 32.5.
\]
\end{proof}

\begin{theorembox}{Variance of the sum}{theo:sum_variance}
\label{theo:sum_variance}
The variance of the sum of a five-card hand drawn without replacement is \cite{feller1968introduction}:
\[
\mathrm{Var}[\Sigma(h)]
= 5 \cdot \sigma_V^2 \cdot \frac{N - 5}{N - 1}
\approx 22.86,
\]
where $N = 32$ and $\sigma_V^2 = 5.25$.
\end{theorembox}

\begin{proof}[Sketch]
For a sample of size $n$ drawn without replacement from a population of size $N$:
\[
\mathrm{Var}[S_n] = n \cdot \sigma^2 \cdot \frac{N-n}{N-1}.
\]
We apply this directly with $n=5$, $N=32$, and $\sigma^2 = 5.25$:
\[
\mathrm{Var}[\Sigma(h)] = 5 \cdot 5.25 \cdot \frac{27}{31} \approx 22.86.
\]
\end{proof}

\begin{propositionbox}{Hypergeometric distribution of hand sums}{prop:hypergeometric_hand_sums}
\label{prop:hypergeometric_hand_sums}
The distribution of hand sums $\Sigma(h)$ arises from a convolution of hypergeometric distributions over the value multiset. For any specific hand configuration, the probability depends on the multivariate hypergeometric law governing the selection of cards without replacement.
\end{propositionbox}

% ========================================
\subsection{Combinatorics of Distributions}
\label{subsec:distribution_combinatorics}

\begin{theorembox}{Total number of distributions}{theo:nb_distributions}
\label{theo:nb_distributions}
The total number of possible distributions for $n$ players, each receiving five cards, is \cite{ross2014first}:
\[
N_{\mathrm{dist}}(n)
= \frac{32!}{(5!)^n \, (32 - 5n)!}.
\]
\end{theorembox}

\begin{proof}
We consider a sequential distribution:
\[
\binom{32}{5},\quad
\binom{27}{5},\quad
\ldots,\quad
\binom{32 - 5(n-1)}{5}.
\]
The product is written as:
\[
\prod_{k=0}^{n-1} \binom{32 - 5k}{5}
= \frac{32!}{(5!)^n (32 - 5n)!}.
\]
\end{proof}

\begin{exemplebox}{Orders of magnitude}{ex:magnitude_orders}
\label{ex:magnitude_orders}
\begin{itemize}
    \item $n=2$: $N_{\mathrm{dist}}(2) = 658,008,000$
    \item $n=3$: $N_{\mathrm{dist}}(3) \approx 2.25 \times 10^{11}$
    \item $n=4$: $N_{\mathrm{dist}}(4) \approx 2.76 \times 10^{14}$
\end{itemize}
\end{exemplebox}

% ========================================
\subsection{Cora Probability}
\label{subsec:cora_probability}

\begin{propositionbox}{Hypergeometric distribution of value 3 cards}{prop:distribution_3}
\label{prop:distribution_3}
The probability of receiving exactly $k$ cards of value 3 follows a hypergeometric distribution (sampling without replacement):
\[
P(X = k)
= \frac{\binom{4}{k} \binom{28}{5-k}}{\binom{32}{5}},
\qquad k \in \{0,1,2,3,4\}.
\]
\end{propositionbox}

\begin{proof}
This is a direct application of the hypergeometric distribution: there are 4 cards of value 3 in the deck of 32 cards, and we draw 5 cards without replacement.
\end{proof}

\begin{exemplebox}{Numerical values}{ex:numerical_values_3}
\label{ex:numerical_values_3}
\[
\begin{aligned}
P(X=0) &\approx 0.488,\\
P(X=1) &\approx 0.411,\\
P(X=2) &\approx 0.095,\\
P(X=3) &\approx 0.006,\\
P(X=4) &\approx 0.0001.
\end{aligned}
\]
\end{exemplebox}

% ========================================
\subsection{Probability of Immediate Victory by Triple 7}
\label{subsec:triple7_probability}

\begin{propositionbox}{Distribution of value 7 cards}{prop:distribution_7}
\label{prop:distribution_7}
Let $Y$ be the number of value 7 cards in a hand (sampling without replacement):
\[
P(Y = k)
= \frac{\binom{4}{k} \binom{28}{5-k}}{\binom{32}{5}},
\qquad k \in \{0,1,2,3,4\}.
\]
\end{propositionbox}

\begin{propositionbox}{Probability of Triple 7}{prop:triple7_probability}
\label{prop:triple7_probability}
The probability of obtaining exactly three cards of value 7 is:
\[
P(Y = 3)
= \frac{\binom{4}{3} \binom{28}{2}}{\binom{32}{5}}
\approx 0.00751.
\]
\end{propositionbox}

\begin{proof}
We note that:
\[
\binom{4}{3} = 4,\qquad
\binom{28}{2} = 378,\qquad
\binom{32}{5} = 201376.
\]
Thus:
\[
P(Y=3) = \frac{4 \cdot 378}{201376}
= \frac{1512}{201376}
\approx 0.00751.
\]
\end{proof}

\begin{notebox}{Rarity of Triple 7}{note:triple7_rarity}
\label{note:triple7_rarity}
We observe that the "triple 7" condition is approximately \textbf{13 times rarer} than the $\Sigma(h)\leq 21$ condition, but it remains observable over long sessions.
\end{notebox}

\begin{propositionbox}{Expected number of Triple 7s over $T$ games}{prop:triple7_expectation}
\label{prop:triple7_expectation}
\[
\mathbb{E}[\text{Triple 7}] = T \cdot P(Y = 3).
\]
\end{propositionbox}

\begin{exemplebox}{Practical case}{ex:triple7_practical_case}
\label{ex:triple7_practical_case}
For $T = 100$ games:
\[
\mathbb{E}[\text{Triple 7}] \approx 0.75.
\]
We therefore expect to see a triple 7 on average every $\approx 133$ games.
\end{exemplebox}

% ========================================
\subsection{Combined Probability of Immediate Victory}
\label{subsec:combined_victory_probability}

\begin{theorembox}{Total probability of immediate victory}{theo:total_victory_probability}
\label{theo:total_victory_probability}
\[
P(\text{Immediate Victory})
= P(\Sigma(h)\leq 21)
+ P(Y=3)
- P(\Sigma(h)\leq 21 \cap Y=3).
\]
\end{theorembox}

\begin{notebox}{Intersection of events}{note:victory_intersection}
\label{note:victory_intersection}
We recognize that the two events are not disjoint: a hand containing three 7s and two very weak cards could satisfy both conditions.
However, we show that this intersection is theoretically zero.
\end{notebox}

\begin{propositionbox}{Null intersection}{prop:null_intersection}
\label{prop:null_intersection}
\[
P(\Sigma(h)\leq 21 \cap Y=3) = 0.
\]
\end{propositionbox}

\begin{proof}
Three cards of value 7 already sum to 21.
The two remaining cards would need to have a sum $\leq 0$ to satisfy $\Sigma(h)\leq 21$, which we know is impossible.
\end{proof}

\begin{corollairebox}{Simplified total probability}{cor:simplified_victory_probability}
\label{cor:simplified_victory_probability}
\[
P(\text{Immediate Victory})
\approx P(\Sigma(h)\leq 21) + 0.75\%.
\]
If we estimate $P(\Sigma(h)\leq 21)\approx 1\%$, then:
\[
P(\text{Immediate Victory}) \approx 1.75\%.
\]
\end{corollairebox}