% ========================================
% CHAPTER 2: MATHEMATICAL FRAMEWORK
% ========================================

\section{Mathematical Framework of the \textbf{F5 Game}}
\label{chap:math_framework}

This chapter establishes the fundamental mathematical structure of the \textbf{F5 Game}. The objects manipulated — cards, players, hands, game states — are defined axiomatically to allow for the formal statement and rigorous proof of the game's properties in the following chapters.

% ========================================
\subsection{Fundamental Sets}
\label{subsec:fundamental_sets}

\begin{definitionbox}{Set of values}{def:values}
\label{def:values}
The set of possible values for a card is:
\[
V = \{3,4,5,6,7,8,9,10\} \subset \N,
\qquad |V| = 8.
\]
\end{definitionbox}

\begin{definitionbox}{Set of suits}{def:suits}
\label{def:suits}
The set of suits (colors) is:
\[
S = \{\heartsuit,\clubsuit,\diamondsuit,\spadesuit\},
\qquad |S| = 4.
\]
\end{definitionbox}

\begin{definitionbox}{Vernacular names of suits}{def:suit_names}
\label{def:suit_names}
Each suit $s \in S$ is associated with a common name in the game:
\[
\text{name}(s) \in \{\text{Zin},\text{Tchaka},\text{Coubi},\text{Black}\}.
\]
\begin{itemize}
    \item $\diamondsuit$: \textbf{Zin}
    \item $\clubsuit$: \textbf{Tchaka}
    \item $\heartsuit$: \textbf{Coubi}
    \item $\spadesuit$: \textbf{Black}
\end{itemize}
\end{definitionbox}

\begin{notebox}{Usage of vernacular names}{note:suit_names}
\label{note:suit_names}
We use these names in oral announcements, particularly when the controller of a round chooses the required suit.
\end{notebox}

\begin{definitionbox}{Card space}{def:card_space}
\label{def:card_space}
The card space is the Cartesian product \cite{grimaldi2003discrete}:
\[
\C = V \times S = \{(v,s) : v \in V,\ s \in S\}.
\]
The full deck is:
\[
D = \C, \qquad |D| = 32.
\]
\end{definitionbox}

% ========================================
\subsection{Players and Structural Parameters}
\label{subsec:players_parameters}

\begin{definitionbox}{Set of players}{def:players}
\label{def:players}
The set of players is:
\[
\mathcal{N} = \{1,2,\ldots,n\},
\qquad n \in \{2,3,4\}.
\]
\end{definitionbox}



\begin{definitionbox}{Dealer}{def:dealer}
\label{def:dealer}
The dealer $d \in \mathcal{N}$ is the winner of the previous game. For the first game of a session, we choose $d$ randomly or by convention.
\end{definitionbox}



\begin{definitionbox}{Direction of play}{def:play_direction}
\label{def:play_direction}
The direction of play is a variable:
\[
\text{direction} \in \{\text{clockwise},\text{counter-clockwise}\},
\]
determining the order of application of the circular permutation $\sigma$ on $\mathcal{N}$.
\end{definitionbox}

\begin{definitionbox}{Initial stake}{def:initial_stake}
\label{def:initial_stake}
The initial stake is a strictly positive real number:
\[
M_0 \in \R_+^*.
\]
\end{definitionbox}

\begin{definitionbox}{Number of rounds}{def:nb_rounds}
\label{def:nb_rounds}
The number of rounds in a game is constant:
\[
R = 5.
\]
\end{definitionbox}

% ========================================
\subsection{Dealing and Hands}
\label{subsec:dealing_hands}

\begin{definitionbox}{Player's hand}{def:player_hand}
\label{def:player_hand}
A player $i$'s hand is a subset \cite{grimaldi2003discrete}:
\[
h_i \subseteq D.
\]
\end{definitionbox}

\begin{definitionbox}{Valid distribution}{def:valid_distribution}
\label{def:valid_distribution}
A valid distribution is a partition:
\[
\mathcal{D} = (h_1,\ldots,h_n,h_{\text{rest}})
\]
such that:
\begin{enumerate}
    \item $\bigcup_{i=1}^n h_i \cup h_{\text{rest}} = D$;
    \item $h_i \cap h_j = \emptyset$ for $i \neq j$;
    \item $|h_i| = 5$ for all $i$;
    \item $|h_{\text{rest}}| = 32 - 5n$.
\end{enumerate}
\end{definitionbox}

\begin{definitionbox}{Cutting the deck}{def:deck_cut}
\label{def:deck_cut}
The first player to receive cards chooses:
\begin{itemize}
    \item \textbf{Pass:} no cut;
    \item \textbf{Cut:} choice of an integer $k \in \{1,\ldots,32\}$, defining:
    \[
    D_{\text{top}} = (c_1,\ldots,c_k),\qquad
    D_{\text{bottom}} = (c_{k+1},\ldots,c_{32}).
    \]
\end{itemize}
\end{definitionbox}

\begin{definitionbox}{Dealing procedure}{def:dealing_procedure}
\label{def:dealing_procedure}
Dealing is a deterministic application producing a valid distribution, according to the rules:
\begin{enumerate}
    \item each player first receives 3 cards;
    \item then 2 additional cards;
    \item if $D_{\text{bottom}}$ is exhausted, we concatenate $D_{\text{top}}$.
\end{enumerate}
\end{definitionbox}

% ========================================
\subsection{Fundamental Functions}
\label{subsec:fundamental_functions}

\begin{definitionbox}{Value function}{def:value_function}
\label{def:value_function}
The value function is the canonical projection:
\[
\text{val} : \C \to V,\qquad \text{val}((v,s)) = v.
\]
\end{definitionbox}

\begin{definitionbox}{Suit function}{def:suit_function}
\label{def:suit_function}
The suit function is the second projection:
\[
\text{col} : \C \to S,\qquad \text{col}((v,s)) = s.
\]
\end{definitionbox}

\begin{definitionbox}{Hand sum}{def:hand_sum}
\label{def:hand_sum}
The sum of the values of a hand $h$ is:
\[
\Sigma(h) = \sum_{c \in h} \text{val}(c).
\]
\end{definitionbox}

\begin{notebox}{Properties of projections}{note:projection_properties}
\label{note:projection_properties}
The functions $\text{val}$ and $\text{col}$ are total, deterministic, surjective, and non-injective. They constitute the natural projections \cite{grimaldi2003discrete} of the Cartesian product $V \times S$.
\end{notebox}

% ========================================
\subsection{Game State}
\label{subsec:game_state}

\begin{definitionbox}{Instantaneous game state}{def:game_state}
\label{def:game_state}
A game state at time $t$ is a tuple:
\[
G(t) = (H(t), r(t), c(t), s_r, P(t), \sigma, \text{direction}, d, M_0, \text{Disq}(t)),
\]
where:
\begin{itemize}
    \item $H(t)$: vector of hands;
    \item $r(t)$: current round;
    \item $c(t)$: controller;
    \item $s_r$: requested suit;
    \item $P(t)$: cards played;
    \item $\sigma$: circular permutation;
    \item $\text{direction}$: direction of play;
    \item $d$: dealer;
    \item $M_0$: initial stake;
    \item $\text{Disq}(t)$: disqualified players.
\end{itemize}
\end{definitionbox}

\begin{propositionbox}{Markovian round evolution}{prop:markov_property}
\label{prop:markov_property}
The sequence of game states $(G(t))_{t=1}^5$ forms a finite, non-stationary Markov chain. That is, for any round $r \in \{1,2,3,4\}$:
\[
P(G(r+1) \mid G(r), G(r-1), \ldots, G(1)) = P(G(r+1) \mid G(r)).
\]
\end{propositionbox}

\begin{proof}
The state $G(r)$ contains all information needed to determine the distribution of $G(r+1)$:
\begin{itemize}
    \item The current hands $H(r)$ determine the available cards;
    \item The controller $c(r)$ determines who chooses the suit;
    \item The history of played cards is included in $G(r)$.
\end{itemize}
Past states $G(1), \ldots, G(r-1)$ provide no additional information once $G(r)$ is known. This is precisely the Markov property.
\end{proof}

% ========================================
\subsection{Compatibility and Playability}
\label{subsec:compatibility_playability}

\begin{definitionbox}{Compatibility set}{def:compatibility_set}
\label{def:compatibility_set}
For a player $i$ and a requested suit $s_r$:
\[
\mathcal{S}_{i,r} = \{c \in h_i(r) : \text{col}(c) = s_r\}.
\]
\end{definitionbox}

\begin{definitionbox}{Legal playability}{def:legal_playability}
\label{def:legal_playability}
A card $c$ is legally playable if:
\[
\mathcal{S}_{i,r} = \emptyset
\quad\text{or}\quad
c \in \mathcal{S}_{i,r}.
\]
\end{definitionbox}

% ========================================
\subsection{Winner Determination}
\label{subsec:winner_determination}

\begin{definitionbox}{Valid cards for victory}{def:valid_cards}
\label{def:valid_cards}
The valid cards for round $r$ are:
\[
\mathcal{V}_r = \{c \in P(r) : \text{col}(c) = s_r\}.
\]
\end{definitionbox}

\begin{definitionbox}{Winner function}{def:winner_function}
\label{def:winner_function}
The winner of round $r$ is:
\[
w(r) = \arg\max_{i : c_{i,r} \in \mathcal{V}_r} \text{val}(c_{i,r}).
\]
\end{definitionbox}

% ========================================
\subsection{Immediate Victory Conditions}
\label{subsec:immediate_victory}

\begin{definitionbox}{Triple 7 condition}{def:triple7_condition}
\label{def:triple7_condition}
A hand $h_i$ satisfies the "triple 7" condition if and only if:
\[
|\{c \in h_i : \text{val}(c) = 7\}| = 3.
\]
\end{definitionbox}

\begin{propositionbox}{Immediate victory by triple 7}{prop:triple7_victory}
\label{prop:triple7_victory}
If a player possesses a hand verifying the triple 7 condition, they immediately win the game, independently of other hands.
\end{propositionbox}

\begin{notebox}{Remark on immediate victory}{note:immediate_victory}
\label{note:immediate_victory}
We consider this condition as an alternative to the $\Sigma(h_i) \leq 21$ rule.
\end{notebox}

% ========================================
\subsection{Payment System}
\label{subsec:payment_system}

\begin{definitionbox}{Standard gain}{def:standard_gain}
\label{def:standard_gain}
\[
G_{\text{std}}(W) = (n-1)M_0.
\]
\end{definitionbox}

\begin{definitionbox}{Simple Cora}{def:simple_cora}
\label{def:simple_cora}
If $\text{val}(c_{W,5}) = 3$, then:
\[
G_{\text{cora}}(W) = 2(n-1)M_0.
\]
\end{definitionbox}

\begin{definitionbox}{Double Cora}{def:double_cora}
\label{def:double_cora}
If $\text{val}(c_{W,5}) = 3$ and $\text{val}(c_{W,4}) = 3$, then:
\[
G_{\text{double}}(W) = 4(n-1)M_0.
\]
\end{definitionbox}

\begin{definitionbox}{Payoff vector}{def:payoff_vector}
\label{def:payoff_vector}
The payoff vector is:
\[
\Delta = (\delta_1,\ldots,\delta_n) \in \R^n,
\qquad
\sum_{i=1}^n \delta_i = 0.
\]
\end{definitionbox}

% ========================================
\subsection{Mathematical Interest}
\label{subsec:math_interest}

The \textbf{F5 Game} presents a rich mathematical structure, allowing us to establish formal results \cite{von1944theory} such as:
\begin{itemize}
    \item the impossibility of certain configurations (Theorem~\ref{theo:impossible_config});
    \item the guaranteed uniqueness of the winner of a round (Theorem~\ref{theo:winner_uniqueness});
    \item the strict conservation of the total stake (Theorem~\ref{theo:zero_sum});
    \item a controlled algorithmic complexity (Theorem~\ref{theo:complexity_summary});
    \item the probabilistic analysis of immediate victory conditions.
\end{itemize}